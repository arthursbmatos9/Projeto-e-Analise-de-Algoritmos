\documentclass[12pt]{article}
\usepackage[utf8]{inputenc}
\usepackage[brazil]{babel}
\usepackage{amsmath}
\usepackage{graphicx}
\usepackage{geometry}
\geometry{a4paper, margin=2.5cm}

\title{A Importância da Disciplina de Projeto e Análise de Algoritmos na Era da Inteligência Artificial e da 4ª Revolução Industrial}
\author{Arthur de Sá Braz de Matos}
\date{02/06/2025}

\begin{document}

\maketitle

\section*{Introdução}

Vivemos a 4ª Revolução Industrial, caracterizada pela convergência entre tecnologias como Inteligência Artificial (IA), Internet das Coisas (IoT) e Big Data. Nesse contexto, a computação se torna elemento fundamental não apenas na tecnologia, mas também em áreas tradicionais como Física e Química, sendo parte integrante de descobertas premiadas com o Nobel.

A inteligência, conforme apresentada na palestra, envolve habilidades como:
\begin{itemize}
    \item Processamento de informação,
    \item Raciocínio lógico,
    \item Resolução de problemas,
    \item Aprendizado e memória,
    \item Criatividade e adaptabilidade.
\end{itemize}

Essas capacidades estão diretamente ligadas aos conceitos ensinados na disciplina de Projeto e Análise de Algoritmos, onde se busca desenvolver soluções eficientes para problemas complexos, otimizando o uso de tempo e recursos computacionais.

\section*{Relações com os Tópicos da Disciplina}

\subsection*{1. Otimização e Escolha de Algoritmos}

A busca por soluções inteligentes envolve selecionar o algoritmo mais adequado para determinado problema. Métodos gulosos, divisão e conquista ou programação dinâmica são paradigmas ensinados na disciplina que refletem diferentes abordagens para alcançar eficiência e inteligência computacional.

\subsection*{2. Resolução de Problemas Complexos}

A inteligência descrita na palestra se expressa no contexto computacional pela capacidade de resolver problemas difíceis com estratégias eficientes. Algoritmos que reduzem a complexidade de tempo são essenciais, especialmente na IA, onde grandes volumes de dados são processados.

\subsection*{3. Desafios da IA e a Eficiência Algorítmica}

A palestra aponta dificuldades como:
\begin{itemize}
    \item Obtenção de dados especializados,
    \item Alto custo computacional,
    \item Necessidade de avaliação especializada.
\end{itemize}

Esses desafios estão ligados diretamente à necessidade de algoritmos otimizados, que consumam menos recursos e sejam capazes de operar com infraestruturas limitadas, uma realidade comum em universidades e centros de pesquisa brasileiros.

\subsection*{4. Oportunidades de Pesquisa em Algoritmos}

Apesar da limitação de infraestrutura, a palestra sugere que a criatividade e o conhecimento profundo em algoritmos ainda nos permitem competir em termos de pesquisa. Desenvolver novas heurísticas, estratégias de aproximação ou métodos híbridos pode ser uma contribuição valiosa, mesmo sem grandes datacenters.

\section*{Conclusão}

A disciplina de Projeto e Análise de Algoritmos fornece as bases para a inteligência computacional que está no centro da transformação digital atual. Aprofundar-se nos princípios de algoritmos eficientes, entender suas limitações e propor melhorias é essencial não só para a pesquisa, mas para a aplicação prática da computação nas mais diversas áreas do conhecimento.

\end{document}
